
 As \cite{fdInColorImages} also mentions, the mouth area contain much stronger $C_{r}$ than $C_{b}$ values compared to the other parts of a face. This is due to the fact that the mouth is more often red than blue. By studying combinations of the $C_{r}$ and $C_{b}$ channels, one can therefore realize that the mouth region receives a low response by $\frac{C_{r}}{C_{b}}$ and a high response for $C_{r}^2$. As of such, it is possible to extract an accurate \textit{mouth map} through Equation \ref{eq:mouthMap} and \ref{eq:nn}.

 % component and weaker blue chrominance component than other areas on the face. We can also observe that the mouth region has strong values for the $C_r^2$ component and low values for the $\frac{C_r}{C_b}$ component. Through this reasoning the mouth map is constructed according to Equation \ref{eq:mouthMap} and \ref{eq:nn}.

\begin{equation} \label{eq:mouthMap}
\begin{split}
mouthMap = C_r^2 \cdot (C_r^2 - \eta \cdot \frac{C_r}{C_b})^2
\end{split}
\end{equation}

\begin{equation} \label{eq:nn}
\begin{split}
\eta = 0.95 \cdot \frac{\frac{1}{n} \cdot \sum\limits_{} C_r^2}{\frac{1}{n} \cdot \sum\limits_{} \frac{C_r}{C_b}}.
\end{split}
\end{equation}
The sums in Equation \ref{eq:nn} are defined over all pixels in the image for each component. 
\intendent All color components in Equation \ref{eq:eyeMapChroma}, \ref{eq:eyeMapLuma}, \ref{eq:mouthMap} and \ref{eq:nn} are normalized to the range $[0, 255]$.

%\newline

% =======
% As for the chrominance components in the eye map, we can make a similar observations for the colors at the mouth area. As $\cite{fdInColorImages}$ mentions, we can observe that the mouth area contain much stronger red chrominance component and weaker blue chrominance component than other areas on the face. We can also observe that the mouth region has strong values for $C_r^2$ component and low values for $\frac{C_r}{C_b}$ component. The mouth map is therefore designed as
% \newline
% \newline
% \begin{equation}
% MouthMap = (C_r^2 \cdot (C_r^2 - \eta \cdot \frac{C_r}{C_b})^2
% >>>>>>> 05da8faa2dbc6071a6590e65148c469b28a75a03




% \begin{equation} \label{eq:nn}
% \begin{split}
% \eta = 0.95 \cdot \frac{\frac{1}{n} \cdot \sum\limits_{(x,y) \in M} C_r(x,y)^2}{\frac{1}{n} \cdot \sum\limits_{(x,y) \in M} \frac{C_r(x,y)}{C_b(x,y)}}.
% \end{split}
% \end{equation} 



% $\eta$ is a parameter for the ratio of the average of $C_r^2$ to the average $\frac{C_r}{C_b}$ and $n$ is the number of pixels in the mask. The products $C_r^2$ and $\frac{C_r^2}{C_b^2}$ are normalized to an interval of $[0,255]$. FIGURE!!!!

% =======
% \newline
% Where $\eta$ is a parameter as the ratio of the average of $C_r^2$ to the average $\frac{C_r}{C_b}$ and $n$ is the number of pixels in the mask. The products $C_r^2$ and $\frac{C_r^2}{C_b^2}$ are normalized to an interval of $[0,255]$. FIGURE!!!!
% >>>>>>> 05da8faa2dbc6071a6590e65148c469b28a75a03
