Eyes can be located by recognizing signature eye regions in images. One way to due this is through the utilization of the \textit{YCbCr} color space \cite{fdInColorImages}. This process involves the development of a so called \textit{eye map}, which essentially is a gray scale image that is brighter where the locations of the eyes could be present. The eye map itself is derived from two independent components, the \textit{chroma} component and the \textit{luma} component. After these two components have been compiled they are merged via an AND operation to establish the eye map. 

The interesting parts about the YCbCr color space which allows us to proceed this way are that the $C_{b}$ and $C_{r}$ channels are observed to have high variation around the eyes with the addition that eyes most often contain both dark and bright pixels in the Y channel. Because of these observations, as described in \cite{fdInColorImages}, we can construct  the chroma component of the eye map through Equation \ref{eq:eyeMapChroma} and the luma component by Equation \ref{eq:eyeMapLuma}, where $\tilde{C_r}$ is the inverted $C_r$ channel and where $\oplus$ represents the dilation and $\ominus$ the erosion of the $Y$ channel by a kernel $K_{i}$. The kernel $K_{i}$ can be of various shapes, e.g. an ellipse, however in this report is a circular disk used, where the kernel width is proportional to $i$.

\begin{equation} \label{eq:eyeMapChroma}
\begin{split}
eyeMapChroma = \frac{1}{3} \lbrace C_b^2 + \tilde{C_r}^2 + \frac{C_b}{C_r} \rbrace
\end{split}
\end{equation}

\begin{equation} \label{eq:eyeMapLuma}
\begin{split}
  eyeMapLuma = \sum_{i=1}^{10}\frac{Y\oplus K_{i}}{i + Y\ominus  K_{i}}
\end{split}
\end{equation}




% $C_b$ is the blue chroma component and $\tilde{C_r}$ is the inverted $C_r$ $(255 - C_r)$. The terms $C_b^2$, $\tilde{C_r^2}$ and $\frac{C_b}{C_r}$ are normalized to the range $[0, 255]$.

% To estimate the second component, the luminance, we take advantage of the intensity structure area around the eyes in the luminance component, where the pixels are dark and bright. The morphological operations Dilation and Erosion are used with to highlight the area. The kernels are applied iteratively with different size according to Equation \ref{eq:eyeMapLuma}.



% Y is the luminance component, $K_n$ is the kernel with size $n$ and $ n \in [1,10]$. Before the eye maps are combined, the contrast in the eye map chroma component is enhanced by histogram equalization. Finally, the eye maps are combined with the AND operation: $EyeMap = (EyemapC) \textbf{AND} (EyeMapLuma)$. For polarization and better result, facial areas can be suppressed by using dilation and erosion operations. 






% The facial feature detection is an important step to extract a final face mask. As $\cite{fdInColorImages}$ motivates, eyes and mouth are dominant. %features in the research field to perform face detection. ....The maps are estimated in the ycbcr color space
% \newline
% \indent To extract the eyes, two independent eye maps are built. The first is from chrominance components and the second is from luminance components. The two components are then merged with AND-operator to get a final eye mask. The interesting parts in the chrominance component are the red and blue chroma component. These components are observed to have high variation around the eyes for the blue chroma and higher variation for the red chroma component. This observation is important at the choice of the color space. As described in \cite{fdInColorImages}, we construct the eye map of the first component, the chroma component, by Equation $\ref{eq:eyeMapChroma}$
% \begin{equation} \label{eq:eyeMapChroma}
% \begin{split}
% eyeMapChroma = \frac{1}{3} \lbrace C_b^2 + \tilde{C_r^2} + \frac{C_b}{C_r} \rbrace
% \end{split}
% \end{equation}
% , where $C_b$ is the blue chroma component and $\tilde{C_r}$ is negative $C_r$ (255 - $C_r$). The terms $C_b^2$, $\tilde{C_r^2}$ and $\frac{C_b}{C_r}$ are normalized to the range $[0, 255]$.
% \newline
% \newline
% To estimate the second component, the luminance, we take advantage of the intensity structure area around the eyes in the luminance component, where the pixels are dark and bright. The morphological operations Dilation and Erosion are used with to highlight the area.  The kernels are applied iteratively with different size, see Equation $\ref{eq:eyeMapLuma}$
% \begin{equation} \label{eq:eyeMapLuma}
% \begin{split}
%   eyeMapLuma = \sum_{i=1}^{10}\frac{Y\oplus K_{i}}{i + Y\ominus  K_{i}}
% \end{split}
% \end{equation}
% where Y(x,y) is the pixel value in the luminance component, $K_n$ is the kernel with size $n$ and $ n \in [1,10]$.
% \newline
% Before the eye maps are combined, the contrast in the eye map chroma component is enhanced by histogram equalization. Finally, the eye maps are combined with the AND operation: $eyeMap = and(eyeMapChroma, \ eyeMapLuma)$. For polarization and better result, facial areas can be suppressed by using dilation and erosion operations.



