\begin{figure}[H]
\centering

\begin{subfigure}{.33\textwidth}
  \centering
  \includegraphics[width=0.95\textwidth]{img/fd2/EyeMapChroma.png}
  \caption{}
  % \label{fig:sub1}
\end{subfigure}%
\begin{subfigure}{.33\textwidth}
  \centering
  \includegraphics[width=0.95\textwidth]{img/fd2/EyeMapLuma.png}
  \caption{}
  % \label{fig:sub1}
\end{subfigure}%
\begin{subfigure}{.33\textwidth}
  \centering
  \includegraphics[width=0.95\textwidth]{img/fd/OriginalEyeMap.png}
  \caption{}
  % \label{fig:sub1}
\end{subfigure}%

\begin{subfigure}{.33\textwidth}
  \centering
  \includegraphics[width=0.95\textwidth]{img/fd/OverSaturatedMask.png}
  \caption{}
  % \label{fig:sub1}
\end{subfigure}%
\begin{subfigure}{.33\textwidth}
  \centering
  \includegraphics[width=0.95\textwidth]{img/fd/FilteredFaceMaskEyesReal.png}
  \caption{}
  % \label{fig:sub1}
\end{subfigure}%
\begin{subfigure}{.33\textwidth}
  \centering
  \includegraphics[width=0.95\textwidth]{img/fd/EyeMap2.png}
  \caption{}
  % \label{fig:sub1}
\end{subfigure}%

\caption{Process of computing the eyes locations. (a) is the chromatic component of the eye map, (b) the luminance component, (c) the original eye map, (d) the mask of over saturated regions, (e) the eye candidates while (f) is the final eye map.}
\label{fig:eyeMap}
\end{figure}