This paper describes how to implement a software for face recognition in MATLAB. The implementation should from an input image of an unknown face, detect the face, extract relevant features from it and make a decision if the face in the image belongs to a known person in a database and, if so, to whom? Furthermore the face recognition should work on input images that are modified with a small rotation, scaling or change of tone values.

Some prerequisites were that the faces in the images all looked into the camera and their skin color did not differ a lot from each other. Also the faces were normalized to have the eyes and mouths aligned across all the images.

Face recognition requires solutions to multiple sub problems, such as pre-processing, face detection and feature extraction.

\begin{itemize}
  \item ---kanske något om preprocess---
  \item The face detection makes a couple of operations to find the face of the input image. Some examples of operations are: Lighting compensation, Color Space Transformation, Skin color detection.
  \item The feature extraction finds the eyes and the mouth. It also makes a face boundary detection.
  \item The face recognition part can be made using different methods. In this paper the methods \emph{LPQ}\footnote{LPQ - Local phase quantization} and \emph{Eigenfaces} will be described.
\end{itemize}
